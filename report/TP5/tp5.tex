\chapterimage{./Pictures/cover-table_of_contents}
\chapter{TP5}
    \section{Exercice 3}
        \paragraph{1.}
            C'est possible avec l'option "-m limit --limit 1/h"
        \paragraph{2.}
            --match permet aussi de faire correspondre d'autre éléments, comme l'adresse IP de la source, donc oui.
        \paragraph{3.}
            --match permet de vérifier le port source aussi.
        \paragraph{4.}
            -A INPUT -p tcp -m tcp --dport 22 -j LOGDROP
            Applique la police logdrop aux paquets tcp à destination du port 22.
            -A INPUT -p tcp -m tcp --dport 3389 -j DROP
            Applique la police drop aux paquets tcp à destination du port 3389.
            -A LGRDP -p tcp -m limit --limit 5/min -j LOG --log-prefix "Denied RDP : " --log-level 7
            S'il y a plus de 5 paquets TCP, on enregistre l'essai en notant avant "Denied RDP : "
            -A LGRDP -j drop
            -A LOGDROP -p tcp -m tcp --dport 22 -m state --state NEW -m recent --set --name SSH --rsource
            -A LOGDROP -p tcp -m tcp --dport 22 -m state --state NEW -m recent --update -- seconds 60 --hitcount4 --name SSH --rsource -j LOG --log-prefix "SSH SCAN blocked:" --log-level 6

            Le port knocking consiste à envoyer des requêtes sur certains ports dans le bon ordre afin de modifier un pare-feu distant depuis l'extérieur.
